We've seen in part \ref{lbl:foxtarget} that the fox always runs towards some target with constant speed. Furthermore, the rabbit perpetually runs towards its burrow, also with constant speed. Therefore, we can say that a creature $C$ runs towards a target $T$ with a constant speed of $u$. We seek it's velocity in the $x$ direction ($v_x$) and in the $y$ direction ($v_y$).



\begin{figure}[h]
\label{fig:chasing}
\caption{The scenario when a creature C runs towards a target T.}
\centering
\begin{tikzpicture}
 
\coordinate (O) at (-1,0);
\coordinate (A) at (3,0);
\coordinate (B) at (3,2);
\draw[dashed] (O)--(A);
\draw[dashed] (A)--(B);
\draw[-{Latex[width=2mm]}] (O)--(B);

\filldraw (-1,0) circle[radius=1.5pt];
\node[above left=0pt of {(-1,0)}, outer sep=1pt,fill=white] {C};

\filldraw (3,2) circle[radius=1.5pt];
\node[above right=0pt of {(3,2)}, outer sep=1pt,fill=white] {T};


\tkzLabelSegment[below=1pt](O,A){$\Delta_x$}
\tkzLabelSegment[above=2pt](O,B){$\Delta$}
\tkzLabelSegment[right=2pt](A,B){$\Delta_y$}

\tkzMarkRightAngle[size=0.5,opacity=.4](O,A,B)% square angle here

\tkzLabelAngle[pos = 0.85](B,O,A){$\theta$}
\tkzMarkAngle[fill=gray, size=1.2cm, opacity=.4](A,O,B)


\end{tikzpicture}
\end{figure}


Figure \ref{fig:chasing} shows the scene. Using this we denote the angle between the creature and the target by $\theta$ and the distance between them by $\Delta$. We can then easily deduce that

$$ v_x = u\cos{\theta}, $$
$$ v_y = u\sin{\theta}. $$

Furthermore, we see there's no need to compute $\theta$. We can observe that 

$$ \cos{\theta} = \frac{\Delta_x}{\Delta}$$
$$ \sin{\theta} = \frac{\Delta_y}{\Delta}.$$

We can then combine these equations giving our final equations

$$ v_x = u \frac{\Delta_x}{\Delta},$$
$$ v_y = u \frac{\Delta_y}{\Delta}.$$

The advantage of using these equations rather than computing $\theta$ using trigonometry is that we can implement both components in one line using Octave's vector based arithmetic. We can see this in our implementation in Listing \ref{lst:computeVelocity}.

Note, this function calls a function called \texttt{distance}, which simply computes the distance between two points. It is listed as Listing \ref{lst:distance} in Apprendix \ref{ap:distance}.

 \lstinputlisting[label={lst:computeVelocity}, caption={A function to compute the velocity of an entity facing a target with a given speed.}] {../computeVelocity.m}
